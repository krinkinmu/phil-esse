\section{Ранний этап}

Первые программируемые ЭВМ (EDVAC, EDSAC, MARK 1, LEO, UNIVAC, CSIRAC) появляются в 40-х годах, основанные на идеях Алана Тьюринга~\footnote{Алана Тьюринга считают отцом Computer Science, именно его имя носит самая престижная премия в этой области — премия Тьюринга} и Джона фон Неймана они стали первыми универсальными ЭВМ. Универсальными их делала именно возможность изменять программу, что дало старт развитию компьютерной техники в ее современном виде и положило начало программированию, как дисциплине. Программы для первых компьютеров составлялись из набора очень простых инструкций (сохранить значение из заданного регистра в заданную ячейку памяти, извлечь значение из заданной ячейки памяти в регистр, выполнить арифметическое действие над содержимым двух регистров, выполнить логическое действие над содержимым двух регистров)~\cite{Turing}, потому что все эти инструкции необходимо было реализовывать аппаратно. Набор инструкций современных ЭВМ стал гораздо богаче, но суть не изменилась — каждая инструкция, как правило, представляет простое действие.

В производстве первых компьютеров использовались вакуумные трубки для реализации логических элементов и магнитные барабаны для реализации памяти. Они были очень дорогими в эксплуатации, потребляли много энергии и выделяли много тепла. Это и ненадежность электронных компонентов часто приводили к неполадкам. Вычислительная мощность по сравнению с современными компьютерами была просто мизерной.

Не смотря на все это первое поколение ЭВМ нашло свое применение. Первые компьютеры могли производить тысячи арифметических операций в секунду, т. е. на тот момент они были самыми быстрыми вычислительными устройствами. Стоимость первых ЭВМ была огромной и, конечно, впервые их начали применять для военных нужд - для расчета баллистических траекторий и взлома шифров. Однако очень скоро они нашли и коммерческое применение.

Программировались первые ЭВМ в языке машинных команд. Язык машинных команд был свой для каждой машины. Каждая инструкция, которую могла выполнить машина кодировалась некоторым числом. Кодировались программы либо в виде отверстий на перфолентах или перфокартах, либо с помощью кабелей и переключателей.

Не трудно понять, что программирование первых ЭВМ было чрезвычайно трудоемким занятием и не считалось престижным (строго говоря, это даже не было самостоятельной профессией). Поэтому в это время программирование считалось женской работой, в то время как мужчины занимались более престижной разработкой аппаратной части. Примечательно, что теперь, когда программная инженерия стала очень престижной областью деятельности, в ней преобладают именно мужчины~\cite{CSWOMEN}.

\subsection{Подпрограмма}

Подпрограмма — набор инструкций программы, выполняющих определенную задачу и собранный в одну структурную единицу. Набор инструкций входящих в подпрограмму мы будем называть телом подпрограммы. По сути, подпрограммы это первый способ декомпозиции программы. Декомпозиция — универсальный инженерный принцип. Так как человек не может одновременно оперировать слишком большим числом сущностей, он неизбежно должен разбивать сложную задачу на подзадачи~\cite{Miller}.

Первые ЭВМ, как уже было отмечено, применялись для численных расчетов, но инструкций для решения систем линейных уравнений или численного решения дифференциальных уравнений в наборе команд ЭВМ, конечно, не было. Для решения таких сложных задач необходимо комбинировать простые инструкции ЭВМ в сложные программы.  Чтобы бороться со сложностью программ их разбивают на подпрограммы, каждая из которых решает относительно простую задачу и поэтому ее проще разработать и реализовать.

Различают открытые и закрытые подпрограммы. Суть открытых подпрограмм заключается в том, что программист явным образом должен вставить тело подпрограммы в нужное место программы. Т. е. они существуют только на этапе проектирования программы, но не во время выполнения. Для таких подпрограмм не требуется поддержка со стороны ЭВМ.

Закрытые подпрограммы, в отличие от открытых, существуют на этапе исполнения программы. Тело закрытой подпрограммы присутствует в программе в единственном экземпляре, а чтобы использовать его необходимо вызвать подпрограмму — передать ей управление. По завершении подпрограмма должна вернуть управление вызвавшей ее части программы. По сравнению с открытыми подпрограммами закрытые требуют меньше памяти, так как тело закрытых подпрограмм не дублируется, но при этом для них требуется поддержка со стороны ЭВМ.

Понятно, что открытые подпрограммы присутствовали с самого начала программирования, но интересно, что некоторые первые ЭВМ уже проектировались с поддержкой закрытых подпрограмм (инструкции BURRY и UNBURRY в~\cite{Turing},~\cite{Dijkstra:1972:HP}).

Итак, подпрограммы выполняют структурную функцию, позволяют экономить ресурсы ЭВМ, но кроме этого они еще позволяют очевидным образом экономить ресурсы человеческие. Подпрограмму написанную один раз можно использовать многократно. В современном программировании существует огромное количество самых разнообразных библиотек подпрограмм, которые используются в множестве программных проектов и для создания других библиотек подпрограмм. Фактически, главный принцип программной инженерии — не дублировать уже проделанную работу, поддерживается подпрограммами (часто можно встретить фразу «изобретать велосипед», что значит повторять уже проделанную работу).

Возможность не повторять уже проделанную работу дает и другую выгоду — повышение надежности программ. Эта выгода была менее значимой в ранний период ПИ, но ее нельзя переоценить в современной ПИ. При создании программ люди неизбежно допускают ошибки. Нахождение и исправление этих ошибок является естественной частью процесса разработки и поддержки ПО. Библиотеки подпрограмм тестируются многими пользователями (в данном случае в качестве пользователей выступают программные инженеры, которые используют эти библиотеки в своих программных продуктах). Чем дольше используется библиотека тем больше ошибок в ней найдено и исправлено.

Подпрограммы являются исторически первым средством в арсенале инженера-программиста. Они появились вместе с первыми универсальными ЭВМ и существуют и активно используются до сих пор. В~\cite{Dijkstra:1972:HP} Эдсгер Дейкстра называет концепцию подпрограмм первой существенным достижением ПИ. В своей Тьюринговской лекции в 1972 году он говорит, что подпрограммы пережили три поколения ЭВМ и переживут еще несколько. Так и случилось, с момента его лекции прошло более 30 лет, но подпрограммы до сих пор являются ключевой концепцией, которая существует во всех распространенных современных языках программирования.
