\section{Заключение}

В распоряжении современной ПИ находится солидное количество парадигм и методологий. Иногда довольно трудно провести четкие границы между ними. Например, гибкие методологии часто относят в отдельную группу методологий, при этом их отличия от инкрементальных и итеративных методологий довольно расплывчаты. Или, например, объектно-ориентированное программирование и компонентно-ориентированное программирование. Часто выбор между той или иной парадигмой или методологий определяется не рациональными причинами, а модными тенденциями. А обсуждение преимуществ или недостатков одной методологии или парадигмы над другой может стать основанием для "святой войны"~\footnote{Обычно так и говорят - holly war.}.

Есть мнение, что программный кризис, который положил начало ПИ до сих пор не преодолен~\cite{Buettner:SEC}. Кризис ПО даже назвали хроническим~\cite{Gibbs:1994:TCS}. Хотя есть и мнение, что никакого кризиса уже нет~\cite{Colburn:2008:SEC}. Как бы то ни было, у современной ПИ есть проблемы, многие проекты до сих пор не укладываются в сроки и бюджет, а ошибки в ПО скорее правило чем исключение~\cite{Jackson:2011:FSE}.

ПИ сравнительно молодая дисциплина, ее нельзя назвать зрелой~\cite{Wang:2000:CSE}. На разработку ПО можно смотреть и как на науку и как на искусство, но в первую очередь она должна стать инженерной дисциплиной. Как отметил Макконел в~\cite{McConnell:1998:ASE} многие ключевые элементы разработки ПО как инженерной дисциплины уже разработаны, осталось собрать их вместе.
