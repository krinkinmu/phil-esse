\section{Заключение}

Современная ПИ это активная прикладная и исследовательская область деятельности человека. Тело прикладной ПИ инженерии состоит из трех частей.
\begin{itemize}
  \item Высокоуровневые концепции, воплощающиеся в множестве различных форм, например, концепция подпрограммы, концепция ЯВУ. За историю ПИ они претерпели некоторые изменения, например, выделяют пять поколений языков программирования, отличающихся разной степенью абстракции, те языки, которые раньше считались высокоуровневыми (FORTRAN, Algol и др) не идут ни в какое сравнение с современными языками (Java, Scala, Python, C++ и др.) по степени абстракции и выразительности. Концепция подпрограмм так же изменялась со временем от первых простых подпрограмм, решающих одну простую задачу, до мультиметодов, шаблонных функций и др.
  \item Множество парадигм программирования существует в данный момент. В работе были представлены только две из них: структурное программирование, как пример парадигмы имеющей строгие математические основания, и объектно-ориентированное программирование, как пример парадигмы ставшей результатом осмысления практического опыта. Но есть и множество других: функциональное программирование, логическое программирование, метапрограммирование и др.). Парадигмы предлагают набор конкретных средств структурирования программы, которые подходят для решения самых различных задач. Так как ЭВМ применяются повсеместно, диапазон решаемых задач огромен, что и привело к такому разнообразию парадигм. Но парадигмы, как правило, не исключают друг друга. Четко разделить парадигмы по какому-то набору критериев весьма проблематично, пример одной из множества возможных классификаций вы можете найти в~\cite{TPPP}.
  \item Методологии разработки ПО акцентируют свое внимание на организации и планировании процесса разработки, а не на собственно коде программы. Как и с парадигмами, существует огромное количество различных методологий. Классифицировать их можно по разным параметрам: легковесные (Agile) и тяжеловесные («водопад» и его последователи, RUP, спиральная методология и др.), плановые («водопад») и итеративные (спиральная методология, Agile). Но как и парадигмами между различными методологиями может быть довольно трудно провести четкие границы.
\end{itemize}

Знания выработанные ПИ плохо структурированы, из-за чего выбор нужной парадигмы и методологии разработки может стать сложной задачей. Зачастую выбор делается не по рациональным причинам, а в силу популярности той или иной парадигмы и/или технологии, или просто под давлением принятых в компании правил.

Еще один важный момент, который стоит отметить, заключается в том, что не смотря на такое разнообразие практик, до сих пор многие программные проекты не укладываются в сроки и бюджет, т. е. нельзя сказать, что программный кризис 60-х 70-х годов преодолен. Ошибки ПО являются скорее правилом, чем исключением~\cite{Jackson:2011:FSE}. Все это свидетельствует о том, что область ПИ еще нельзя назвать зрелой. В работе~\cite{Wang:2000:CSE} предлагается четырехуровневая модель развития инженерной дисциплины и ПИ оценивается согласно этой модели как молодая дисциплина, находящаяся на переходной стадии между искусством и, собственно, инженерией.
