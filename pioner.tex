\section{Новаторский этап}

Новаторский этап начинается с середины 50-х годов, ПИ как самостоятельно дисциплины все еще не существует, программированием занимаются математики и инженеры электротехники. Нет учебных дисциплин нацеленных на ПИ в целом и программирование в частности. Не существует и понятия управления программными проектами, предсказывать сроки завершения программных проектов практически невозможно. Безсистемный подход к разработке программ этого этапа можно, условно, обозначить как «Code and Fix» - сначала пишется код, потом исправляются проблемы в этом коде, т. е. сначала идет реализация, а потом уже задумываются о требованиях, дизайне, тестировании и тд~\cite{Boehm:1988:SMSD}. Программам не уделяется особого внимания — они просто бесплатное дополнение к ЭВМ. Это довольно курьезное положение дел, ведь ЭВМ без программ бесполезны.

Постепенно на смену ламповым ЭВМ стали приходить транзисторные ЭВМ. Бурное развитие технологий привело к тому, что компьютеры сменяли друг друга каждые пару лет (IBM-701, TRADIAC, Pegasus, Philco-2000, Elliot-803, Siemens-2002, H-1, IBM-7030, Atlas, CDC-6600)~\cite{Glass}. Программирование, в основном сводится, к переписыванию старых программ для новых ЭВМ. В частности и это обстоятельство даст толчек второму существенному прорыву в истории ПИ — языкам программирования высокого уровня.

\subsection{Языки программирования высокого уровня}

Языки программирования высокго уровня (ЯВУ) - формальные языки описания программ, обладающий большей степень абстрактности, чем язык машинных команд~\cite{HLPL}. Это сущестенный момент в развитии ПИ. Спустя несколько лет после создания первых программируемых ЭВМ инженерам приходит осознание очевидной проблемы — программировать в языке машинных команд черезвычайно сложно и потому не продуктивно и чревато ошибками. Язык машинных команд — язык удобный для ЭВМ, но совершенно непригодный для человека. Он существенно ограничивает возможности использования компьютера. Создавать программные системы средней и высокой сложности на языке машинных команд очень сложно.

При проектировании программ с использованием языка машинных команд инженер оперирует такими понятиями как регистры, биты, машинные слова, память. При использовании ЯВУ инженер оперирует такими понятиями как числа, символы, строки, перемнные и константы (именованые значения и участки памяти), появляеся концепция типов. Таким образом ЯВУ изменяет язык, а вместе с ним и категории мысли используемые в ПИ, выводит их несколько более абстрактный уровень.

Можно отметить несколько занчимых языков программирования разработанных во время этого этапа: FORTRAN, LISP, Algol и PL/1. Это не первые ЯВУ, которые были разработаны и использовались, но их можно назвать самыми заметными в ранней истории ПИ. Сейчас, спустя много лет, на некоторые из них, мы смотрим, скорее, как на отрицательные примеры, чем как на положительные, однако отрицательный опыт тоже ценен в ПИ.

\paragraph{FORTRAN} (FORmula TRANslator) — первый ЯВУ получивший действительно широкое распространение. Чтобы подчеркнуть значимость этой разработки, достаточно упомянуть, что язык FORTRAN до сих пор используется, не так широко и активно как раньше — его применения сократилось до использования в научных вычислениях, но это совершенно удивительный срок жизни для технологии (я имею ввиду именно FORTRAN, а не саму идею ЯВУ) в области ПИ. Впрочем не смотря на всю значительность этой разработки, можно констатировать, что язык FORTRAN безнадежно устарел, и как отметил Дейкстра, чем раньше мы забудем о его существовании тем лучше~\cite{Dijkstra:1972:HP}.

\paragraph{LISP} (LISt Processing language), в отличие от FORTRAN, даже спустя много лет рассматривается как положительный опыт в истории ПИ. В начале своего появления, он не пользовался большой популярностью, однако, стоит заметить, что свое применение он все же нашел — в области искусственного интеллекта он был основным языком, вероятно за свою исключительную простоту и выразительность. LISP был совершенно отличен от FORTRAN, и многие современные технологии появились впервые именно в LISP: автоматическое управление памятью (активно используется очень многими современными ЯВУ, например, целой группой языков для JVM, Python, JavaScript и др), динамическая типизация (мнение об этой концепции в среде ПИ не однозначно, есть сторонники такого подхода, есть противники, есть те, кто признают важность и динамической и статической типизации), функции высшего порядка (функции, которые принимают функции, как аргумент или возвращают их как результат своей работы) и др. Как и FORTRAN LISP до сих пор используется (в виде множества различных более современных диалектов, сохраняющих общие концепции с оригинальной разработкой).

ЯВУ сделали код программ понятным человеку и ввели новые понятия в язык ПИ, но кроме того, компиляторы первых языков программирования были первыми действительно большими программными проектами. Именно в области разработки компиляторов впервые стал применяться систематический подход к разработке ПО. Кроме того область языков программирования дала толчек развитию теории формальных языков, в конце новаторского этапа выходит множество статей в этой области (например, Дональд Кнут «On the Translation of Languages from Left to Right», Джей Эрли «An Efficient Context-Free Parsing Algorithm», Джон Бэкус «The syntax and semantics of th proposed international algebraic language of the Zurich ACM-GAMM Conference» и др.)
