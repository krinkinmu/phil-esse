\section{Современное состояние}

Два существенных достижения, которые определяют лицо ПИ на современном этапе ее развития это Интернет в его современном понимании~\footnote{Имеется ввиду использование Интернет в коммерческих целях, которое началось в конце 80-х годов и получило широкое распространение в 90-х, а также появление графических браузеров в начале 90-х.} и еще большая миниатюризация компьютерной техники~\footnote{Имеются ввиду различная носимая электроника - смартфоны, планшеты, плееры и пр.}. Вместе они привели к еще большему возрастанию потребности в ПО~\footnote{Web-сайты, мобильные приложения и тд.}.

Интернет стал новой целевой платформой разработки ПО. Зачастую новые программные проекты нельзя назвать большими и сложными~\footnote{Сложные проекты никуда не делись, более того их стало больше, но вместе с тем появилось очень много небольших программных проектов, например, типичный Web-сайт небольшой организации нельзя назвать сложным проектом.}, более того, многие из них вполне можно назвать типовыми. Для разработки таких проектов не требуется большая команда. Получают распространение различные итеративные инкрементальные модели разработки, объединенные под модным словом "agile".

Благодаря возможности переиспользования готовых компонентов все большее распространение получает объектно-ориентированная парадигма~\footnote{Например, в это время появляются такие языки/технологии как Java, Python, Ruby и др.} и компонентно-ориентированное программирование~\footnote{Идея не новая - приложение собирается из готовых компонент, разработанных независимо и придерживающихся одного интерфейса, поэтому я не буду ее касаться здесь. Новая реализация, появляются такие технологии как EJB, COM, CORBA и другие, которые определяют какой интерфейс должен реализовывать компонент.}.

Интернет, как средство глобальной коммуникации, привел к распространению свободного ПО. Процесс разработки свободного ПО можно назвать эволюционным и он отличен по ряду признаков от известных методологий разработки.

\paragraph{"Базар" - метод разработки свободного ПО}. Началом движения за свободное ПО будем считать 1983 год, когда Ричард Столман написал письмо о начале проекта GNU (GNU is Not Unix). Целью проекта было разработать свободную операционную систему совместимую с UNIX. Это стало результатом коммерциализации UNIX годом ранее. Рассмотрение истории движения за свободное ПО выходит за рамки этой работы, в первую очередь интересен именно подход к разработке свободного ПО с появлением Интернет.

Основная особенность разработки свободного ПО заключается в том, что любой заинтересованный человек может принять участие в разработке программы. Любые изменения, которые он сделает будут оценивать открытым сообществом, и если сообщество разработчиков посчитает их полезными и достаточно качественными, то они будут приняты в проект. Конечно, в проекте присутствует один или несколько человек, которые устанавливают правила проекта, планируют выпуски новых версий и имеют решающее слово при включении в проект новых изменений. Однако направление развития проекта во многом определяется открытым сообществом разработчиков.

Эрик Реймонд в~\cite{Raymond:1999:CB} описывая процесс разработки ядра Linux~\footnote{Одна из нескольких свободных операционных систем семейства Unix.} называет такой процесс разработки "базар". Свободное ПО до появления Интернет разрабатывалось не публично, небольшими группами людей, публично распространялись только конечные результаты разработки. Разработка Linux, напротив, открыта, в ней постоянно участвует множество людей со всего мира со своими интересами и подходами. Как отмечает Реймонд, кажется, что получить стабильное ПО на выходе можно только чудом.

На деле же такой подход, как оказалось, работает очень хорошо. Основная идея работы~\cite{Raymond:1999:CB} заключается в том, что при наличии достаточного количества глаз все ошибки всплывают~\footnote{Это утверждение называют Законом Линуса, Линус Торвальдс - основатель и лидер проекта Linux Kernel.}. Соответственно, чем больше различных людей участвет в разработке свободного ПО тем выше будет качество.

Сейчас существует множество программных проектов которые разрабатываются открыто~\footnote{Например, Linux Kernel, FreeBSD, OpenSSL, GCC, проекты Mozilla и многие другие.}. Многие открытые проекты сейчас активно используются в промышленности, а компании, в свою очередь, часто делают свои проекты открытыми.

Методы разработки открытого ПО сильно отличаются от других методологий своей децентрализованностью и отсутствием планирования. Открытость позволяет любому заинтересованному человеку проверить ПО и, при необходимости, исправить ошибки или добавить новую функциональность~\footnote{Кроме заинтересованности, конечно, требуются еще и серьезные технические знания.}, в то время как закрытому ПО приходится просто доверять.
