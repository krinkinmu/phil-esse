\section{Введение}

Институт инженеров по электротехнике и электронике (IEEE) определяет программную инженерию (ПИ) следующим образом: ПИ это приложение систематического, дисциплинированного, измеряемого подхода к развитию, функционированию и сопровождению программного обеспечения (ПО), а также к исследованию этих подходов, т. е. приложение дисциплины инженерии к программного обеспечению~\cite{IEEEVOC}.

Термин “программная инженерия” был введен Фредирхом Бауэром. В его версии ПИ это установление и использование обоснованных инженерных принципов для экономного получения надежного ПО, которое работает на реальных машинах. Впрочем, годом ранее Бауэр предложил и другое, не вполне серьезное, определение: ПИ это та часть информатики, которая слишком сложна для ученых-информатиков.

А согласно Барри Боему ПИ это практическое приложение научных знаний в разработке и конструировании программ и сопутствующей документации, необходимой для их разработки, функционирования и поддержки~\cite{Boehm:1976:SE}.

Согласно этим определениям ПИ это дисциплина, нацеленная на выработку, исследование и применение методологий разработки ПО в условиях ограниченных ресурсов (время, бюджет) и с надлежащим качеством. Т. е. ПИ есть ни что иное как приложение инженерного дела к ПО, но, конечно, есть ряд особенностей, выделяющих ее среди других инженерных дисциплин:
\begin{itemize}
  \item программы не материальны, т. е. они не имеют массы, объема или других физических свойств, их нельзя измерить привычными инструментами; методы измерения характеристик ПО являются важной областью программной инженерии – метрология ПО;
  \item программы не подвержены износу, чего нельзя сказать об аппаратном обеспечении, все ошибки вызваны исключительно дизайном и реализации программы, они не появляются и не исчезают со временем, хотя возможно, для проявления ошибки потребуется время;
\end{itemize}

Но, вероятно, главное отличие от классических инженерных дисциплин заключается в  подходе к разработке. Как правило инженер обладает набором средств, физических и математических знаний, которые позволяют определить свойства продукта (с некоторой степенью точности), до его непосредственной реализации. В программной инженерии, как правило, таких средств нет, либо их использование не оправдано, поэтому опыт приобретает еще более важное значение.

Опыт важная часть любой инженерной дисциплины, ПИ не стала исключением. Но опыт в ПИ, наверно, даже более важен, чем в других дисциплинах, в силу ограниченной применимости формальных методов для построения программных систем~\cite{McConnell}. Многие представители образования подчеркивают важность получения опыта работы над настоящими большими программными проектами в образовании программного инженера~\cite{Wohlin},~\cite{Kurkovsky}. Опыт и анализ ошибок заставляют инженерные дисциплины, и ПИ в частности, развиваться~\cite{Jackson:2008:ASE}. Многие парадигмы программирования и методологии разработки программных проектов появились как результат переработки предыдущего опыта.

Само появление понятия "программная инженерия" связано с понятием "кризиса программного обеспечения". Кризис ПО - это исторический период, когда был накоплен опыт неудачных проектов достаточный, чтобы осознать необходимость систематического подхода, когда инженеры, наконец, начали задуматься над тем как именно они разрабатывают ПО.

В настоящее время ПИ это активно развивающаяся техническая дисциплина, ее развитие непосредственно связано с развитием компьютерных и телекоммуникационных технологий (Internet, сотовая связь, IoT). Компьютеризированные технологии применяются в медицине, авиации, науке и в быту. Но кроме этого, ПИ это и активная академическая дисциплина с множеством конференций, журналов, образовательных программ и научных исследований. Исследования в области ПИ нацелены на разработку технологий повышающих качество ПО, на повышение эффективности процессов разработки ПО, делающих ее более предсказуемой.

В данной работе излагается история ПИ начиная с ее самых ранних периодов. В работе делается акцент на появлении идей, парадигм и методологий разработки ПО. Мне хотелось показать, как появлялись и развивались эти идеи, реакцией на какие проблемы они стали. В работе рассматриваются самые различные концепции, от абстрактных идей (идея подпрограммы и идея языка программирования), так и более конкретные парадигмы, как имеющие теоретические корни (структурное программирование), так и появившиеся в недрах IT корпораций в результате осмысления прошлого опыта (объектно-ориентированное программирование).

Стоит отметить, что рассмотреть все возможные парадигмы программирования и методологии разработки в одной работе нет никакой возможности - их количество огромно~\cite{PARADIGMS}. Зачастую между различными парадигмами нельзя провести более или менее четкие границы, поэтому даже классификация всех парадигм может стать довольно трудной задачей (например,~\cite{Roy:TPPP} только одна из возможных классификаций, причем не самая полная). Поэтому в данной работе не преследуется цель их полного изложения, я остановлюсь лишь на малой части наиболее существенных парадигм и методологий.

Есть несколько подходов к разделению истории ПИ на этапы, например,~\cite{Glass} и~\cite{Aguila:2014:MSE}. В данной работе используется последняя, в основном по тому, что она покрывает современную историю ПИ, а не только раннюю. В дополнение к описанным этапам я рассмотрю и этап от момента появления первых универсальных ЭВМ. Этот этап совсем не продолжительный во времени, но, я считаю, правильным начать именно с него, так как, во-первых, он является отправной точкой всей ПИ, во-вторых, на этом этапе, может быть не осознавая этого, первые программисты уже применяли некоторые инженерные принципы в своей работе.
